\section{Methodology}

\tblSpecs

We will evaluate the effectiveness of our change using four different systems,
with detailed system specifications listed in \autoref{tbl:spec}:

\textbf{\texttt{stg1}} is a new server-grade system sporting an AMD Ryzen
Threadripper PRO 3975WX CPU. This CPU supports AVX2 instructions and fused
multiply-add (FMA3) instructions.

\textbf{\texttt{stg2}} is a new server-grade system sporting an AMD EPYC 7763 CPU.
This CPU supports AVX2 instructions and fused multiply-add (FMA3) instructions.

\textbf{\texttt{bcn6}} is an older server-grade system sporting an Intel Xeon
E5-2697.
This CPU supports AVX2 instructions and fused multiply-add (FMA3) instructions.
It runs at a substantially lower frequency compared to \texttt{stg1}.

\textbf{\texttt{mac}} is my MacBook Pro 2019 laptop benchmarked while charging.
It sports an Intel Core i5-1038NG7 CPU. This CPU supports both AVX2 and AVX512,
but Rust's compiler does not have stable support for AVX512. Because of this,
we are only using AVX2 instructions.
\texttt{mac} also has a fused multiply-add unit (FMA3).

We use Rust's criterion benchmarking library, which automatically sets the
number of benchmark samples to run in order to ensure statistically significant
results.
