\section{Contribution}

\figOldLayout
\figNewLayout

\autoref{fig:oldLayout} and \autoref{fig:newLayout} illustrate the change of
representation of polynomials that we employed in this work.
The goal is to enable SIMD vectorization.

Specifically, the old representation represented each coefficient as a vector
of residues.
The new representation holds residues corresponding to a specific small modulus
across all coefficients in a single vector, called a residue polynomial.

In fancy math notation, this is a change from

\begin{equation*}
b_i = \sum_{j=0}^k (a_j (M_j \mod p_i) (N_j \mod p_i) \mod p_i
\end{equation*}

into

\begin{align*}
    \begin{bmatrix}
        b_i^{(0)} \\
        b_i^{(1)} \\
        \vdots \\
        b_i^{(7)}
        \end{bmatrix} &= \sum_{j=0}^k \begin{bmatrix}
        a_j^{(0)} \\
        a_j^{(1)} \\
        \vdots \\
        a_j^{(7)}
        \end{bmatrix}(M_j \mod p_i) (N_j \mod p_i) \mod p_i.
\end{align*}

That is, our new representation enables us to compute the same change RNS base
on multiple polynomial coefficients in parallel using SIMD (e.g., 8 in the
example).

The nice part is that this change can be made transparently to all other
modules of the dumb-ckks FHE library: the change is completely contained in
the internals of the FHE polynomial class.

\subsection{Handling Number-theoretic Transforms}

This change makes SIMDifying NTTs somewhat more challenging. Namely, with the
old representation, the NTT could have been applied at once to the entire FHE polynomial.
This allows for SIMD: each multiplication and addition between NTT `twiddles'
and corresponding elements can be done using SIMD by putting different residues
into different SIMD slots.

While we have not added SIMD NTT support, we describe how it can be added, even
with the new layout:
We do NTTs independently, one residue polynomial at a time. While SIMDifying
this is less straight-forward, standard techniques are known~\cite{kral2003simd}.
