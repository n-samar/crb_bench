% dsm: This section has some good points, but it's not an overview. It's mostly a drawn-out descripton of preliminaries. After thinking about this, I'm going to try to describe BitPacker in a single section, because this doesn't get to the point. I also don't see the point in introducing the term "modulus map". I avoided it in Section 2, and don't see why BitPacker makes this any more complicated.

\section{Overview}
\label{sec:overview}

As mentioned, we define a \emph{modulus map} as a mapping between levels and
moduli.
In this section, we show how modulus maps, hardware, and FHE programs can be
decoupled without impacting precision and security.
This allows us to run the \emph{same} FHE program using \emph{many} different
modulus maps, with no impact to precision or security.
We then show the main idea behind \name's modulus maps, and why they are better
than those of RNS-CKKS.
\autoref{sec:levelMgmt} shows how CKKS's level management operations can be
adapted to be flexible enough to support \name's modulus maps, while
\autoref{sec:modulusChain} presents an algorithm we use to create these maps.

\figOverview

\autoref{sec:background} shows that a modulus map impacts the precision
and security of an FHE program. Thus, not all modulus maps are appropriate
for a given program.
Instead, users must specify precision and security constraints that a modulus
map must meet in order to execute an FHE program correctly. We call these
\emph{program constraints} (\autoref{fig:overview}).

In this context, an \emph{FHE program} is a circuit of inputs, outputs, and
homomorphic and level management operations. Each node is label with a level,
without any information about moduli. \autoref{fig:overview} shows this circuit
for the program $x \ast x + x$.

With this setup, any modulus chain that meets program constraints can correctly
run any FHE program that requires those constraints. In other words,
modulus constraints act as a contract between FHE programs and modulus maps.

Specifically, program constraints are:
\emph{(1)} the maximum number of levels between bootstraps ($L_{max}$),
\emph{(2)} the maximum modulus $Q_{max}$ (impacts security),
\emph{(3)} the minimum modulus $Q_{min}$ (impacts the size of encrypted values
the program can handle correctly), and
\emph{(4)} the target scale at each level $L \mapsto T_L$ (impacts precision).

The hardware accelerator further constrains the modulus map by
requiring that each residue modulus fits into a hardware word
(\autoref{fig:overview}).


Given program and hardware constraints, we can derive many modulus maps $L
\mapsto Q_L$.
In fact, RNS-CKKS is an example of a modulus map and we roughly showed how it
can be constructed from program constraints in \autoref{sec:background}.

\paragraph{Modulus map implies scales:}
A modulus map implies a mapping between levels and scales $L \mapsto S_L$.
This is because the scale of a ciphertext at level $L$ is determined by:
\emph{(1)} the target scale at the highest level $L_{max}$, $T_{L_{max}}$ and
\emph{(2)} the moduli at each level between $L_{max}$ and $L$.

This dependence is best illustrated by referring back to the example of
computing $x \ast x + x$ in \autoref{sec:noise}.
Assume the input $x$ is at $L = 1$, just as in \autoref{fig:overview}.
Thus, $x$'s modulus and scale are $Q_1$ and $S_1$, respectively.
The modulus and scale of the output is $Q_0$ and $S_0 = S_1^2/d$, where $d =
Q_1/Q_0$.
That is, $S_0$ is the scale of a ciphertext that results from multiplying two
ciphertexts at level $1$ with scale $S_1$ and then rescaling by $d = Q_1/Q_0$.
Thus, $S_0 = S_1^2Q_0/Q_1$

In general, the scale at level $L$ will be $S_L = S_{L+1}^2Q_L/Q_{L+1}$ (the
scale that results from rescaling the product of two ciphertexts at level $L$).
By applying this relation inductively, we see that all scales can be derived
from the scale at $L_{max}$ ($T_{L_{max}}$) and the moduli $Q_L$.

\paragraph{Meeting constraints:}
We say that a level map meets program constraints if:
\emph{(1)} $T_L/2 < S_L < 2T_L$ for all levels $L$ (i.e., the scales at each
level are roughly what the user expects),
\emph{(2)} $Q_0 > Q_{min}$ (i.e., the modulus map can handle the size of values
expected by the user), and
\emph{(3)} $Q_{L_{max}} < Q_{max}$ (i.e., the program meets the user's security
budget).

A modulus map meets the hardware bitwidth constraint if all its residue moduli
fit into the bitwidth.

\paragraph{\name vs RNS-CKKS:}
A part of a RNS-CKKS modulus map for $T_L = 2^{40}$ (for all $L$), $L_{max} = 6$,
and a 64-bit hardware word is shown in \autoref{fig:levelsRnsCkks}. We also
show the implied scales on the right. As \autoref{sec:background} explains,
RNS-CKKS chooses residue moduli that roughly match the target scale $T_L =
2^{40}$.
This, unfortunately, leaves 24 bits of each hardware word unused (in red).

In contrast, \name uses modulus maps that match the residue modulus size to the
hardware word (\autoref{fig:levels}). The residues in green are as large as
possible while fitting within a hardware word, thus eliminating overhead (in red).
We call these \emph{non-terminal moduli}.
In contrast, the last residue moduli at each level (in blue) are noticably
smaller than a hardware word, leading to some overhead. We call these
\emph{terminal moduli}.

Critically, \name's modulus map in \autoref{fig:levels} \emph{meets the same
modulus and hardware constraints} as RNS-CKKS's modulus map in
\autoref{fig:levelsRnsCkks}.
But by packing data better, \name requires less hardware words to represent
and operate on the same data.
Because \name meets the same user and hardware constraints as RNS-CKKS, it can
run the same programs on the same hardware with the same precision and security
as RNS-CKKS.


\paragraph{Security:}
Different modulus maps that meet the same program constraints need to have the
same amount of security. Modulus maps impact security only through the
largest size of the largest modulus $Q_{L_{max}}$. Program constraints
dictate that $Q_{L_{max}} < Q_{max}$. Thus, all modulus maps will meet the
this security threshold.
