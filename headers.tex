% MICRO STUFF START

\usepackage{mathptmx} % This is Times font

\usepackage{fancyhdr}
\usepackage[normalem]{ulem}
\usepackage[hyphens]{url}
\usepackage[sort,nocompress]{cite}
\usepackage[final]{microtype}
\usepackage[keeplastbox]{flushend}
\usepackage{subfig}
\usepackage{hyphenat} % for \hyp{}, a hyphen that allows linebreaks on the hyphenated words
\usepackage{graphicx}


% Ensure letter paper
\pdfpagewidth=8.5in
\pdfpageheight=11in

%%%%%%%%%%%---SETME-----%%%%%%%%%%%%%
\newcommand{\microsubmissionnumber}{439}
%%%%%%%%%%%%%%%%%%%%%%%%%%%%%%%%%%%%

\fancypagestyle{firstpage}{
  \fancyhf{}
  \renewcommand{\headrulewidth}{0pt}
  \fancyhead[C]{
      \vspace{10pt}
  }
  \fancyfoot[C]{\thepage}
}

\pagenumbering{arabic}

% MICRO STUFF END (except for hyperref usepackage)

\usepackage{amsmath, amssymb}
\usepackage{enumitem}
\usepackage{listings}
\usepackage{comment}
\usepackage{xspace}
\usepackage{xcolor}
% \usepackage{titling}
\usepackage[normalem]{ulem}
\usepackage{wrapfig}
\usepackage{paralist}
\usepackage[sort,nocompress]{cite}
\usepackage{array}
\usepackage{booktabs}

% alex: Useful for marking things up on paper
%\doublespacing

\author{Nikola Samardzic\\
\texttt{nsamar@csail.mit.edu}}
\newcommand{\name}{BitPacker\xspace}
\newcommand{\x}{$\times$\xspace} % dsm: WATCH OUT. \xspace is important here, and may be important in some of the above ones too!

% Python style for highlighting
\lstset{
    language=Python,
    basewidth=0.5em,
    basicstyle=\footnotesize\ttfamily,
    morekeywords={Tensor},
    deletendkeywords={sum},
    keywordstyle=\bf\color{deepblue}\ttfamily,
    stringstyle=\color{deepgreen},
    commentstyle=\color{deepgreen},
    frame=ltbr,
    showstringspaces=false,
    mathescape=true,
    escapeinside={(*}{*)},
    aboveskip=5pt,
    belowskip=0in,
    belowcaptionskip=-5pt,
    captionpos=b,
    numbers=left,
    xleftmargin=1.75em,
    numbersep=7pt
}


\def\figureautorefname{Fig.}
\def\subfigureautorefname{Fig.}
\def\sectionautorefname{Sec.}
\def\subsectionautorefname{Sec.}
\def\algorithmautorefname{Algorithm\xspace}
\def\paragraphautorefname{Sec.}
\def\equationautorefname{Eq.}

\usepackage[T1]{fontenc} % Avoid garbling symbols and accents

\renewcommand\ttdefault{txtt}

% \usepackage[activate={true,nocompatibility},final,tracking=false,kerning=true,spacing=true,shrink=25,stretch=10]{microtype}

% https://tex.stackexchange.com/questions/240141/
\usepackage[T1]{fontenc} % Avoid garbling symbols and accents

% Custom colors
\definecolor{deepblue}{rgb}{0,0,0.4}
\definecolor{deepred}{rgb}{0.6,0,0}
\definecolor{deepgreen}{rgb}{0,0.5,0}

% Paper version -- comment odd/evenhead for submission
\newcommand{\topbanner}{\textbf{DRAFT --- \input{auto_header.tex}}}
\makeatletter
\def\ps@plain{
%  \def\@oddhead{\hbox{}\normalsize\hfil \topbanner \hfil}
%  \def\@evenhead{\hbox{}\normalsize\hfil \topbanner \hfil}
  \def\@oddfoot{\hbox{}\normalsize\hfil \thepage \hfil}
  \def\@evenfoot{\hbox{}\normalsize\hfil \thepage \hfil}
}
\makeatother

% Temporary macros. Comment for submission!
% \newcommand{\note}[1]{{\bf [~NOTE:~#1~]}}
% \newcommand{\fixme}[1]{{\textcolor{red}{{\bf [~FIXME:~#1~]}}}}
% \newcommand{\todo}[1]{\textcolor{red}{{\bf [~TODO:~#1~]}}}
% \newcommand{\tmp}[1]{{\textcolor{red}{#1}}}
% \newcommand{\OK}[1]{\textcolor{green}{#1}}
% Submission...
\newcommand{\OK}[1]{{#1}}

%\newcommand{\nikola}[1]{\textcolor{red}{\textbf{(Nikola: \emph{#1})}}}
%\newcommand{\alex}[1]{\textcolor{olive}{\textbf{(Alex: \emph{#1})}}}


\renewcommand\ttdefault{txtt}

% Python style for highlighting
\lstset{
    language=Python,
    basewidth=0.5em,
    basicstyle=\footnotesize\ttfamily,
    morekeywords={Tensor},
    deletendkeywords={sum},
    keywordstyle=\bf\color{deepblue}\ttfamily,
    stringstyle=\color{deepgreen},
    frame=ltbr,
    showstringspaces=false,
    mathescape=true,
    escapeinside={(*}{*)},
    aboveskip=5pt,
    belowskip=0in,
    belowcaptionskip=-5pt,
    captionpos=b,
    numbers=left,
    xleftmargin=1.75em,
    numbersep=7pt
}

\newcommand{\tpy}[1]{
    \lstinline[columns=fixed]!#1!
    %\lstinline!#1!
}

\graphicspath{{figures/}}

\newcommand{\cmd}[1]{\texttt{#1}}
\hyphenation{log-reg}

\setlist{nosep, topsep=-\parskip+0.2em,leftmargin=*}
\setlist[enumerate]{label*=\arabic*.}

% dsm: Fix citation format
\makeatletter
\def\citepunct{, }
\def\citedash{--}
\makeatother

% \captionsetup[figure]{aboveskip=2pt,belowskip=-12pt}
% \captionsetup[table]{aboveskip=2pt,belowskip=-12pt}
% \captionsetup[subfigure]{aboveskip=2pt,belowskip=-12pt}
% dsm: Somehow, footnotesize is 9pt and small is 8pt. GO FIGURE.
\usepackage{caption}
\captionsetup[table]{textfont={bf,footnotesize}, labelfont={bf,footnotesize}}
\captionsetup[figure]{textfont={footnotesize,bf}, labelfont={footnotesize,bf}} % Requirement is 9pt, bold
\captionsetup[subfloat]{labelfont=footnotesize, textfont=footnotesize}

\makeatletter
\renewcommand{\paragraph}[1]{\noindent {\bf #1}}
\makeatother

\def\figureautorefname{Fig.}
\def\subfigureautorefname{Fig.}
\def\sectionautorefname{Sec.}
\def\subsectionautorefname{Sec.}
\def\algorithmautorefname{Algorithm\xspace}
\def\paragraphautorefname{Sec.}
\def\equationautorefname{Eq.}

\renewcommand{\topfraction}{1.0}        % max fraction of floats at top
\renewcommand{\bottomfraction}{0.8}     % max fraction of floats at bottom
%   Parameters for TEXT pages (not float pages):
\setcounter{topnumber}{5}
\setcounter{bottomnumber}{5}
\setcounter{totalnumber}{4}     % 2 may work better
\setcounter{dbltopnumber}{5}    % for 2-column pages
\renewcommand{\dbltopfraction}{0.9}     % fit big float above 2-col. text
\renewcommand{\textfraction}{0}      % allow minimal text w. figs
%   Parameters for FLOAT pages (not text pages):
\renewcommand{\floatpagefraction}{0.9}  % require fuller float pages
% N.B.: floatpagefraction MUST be less than topfraction !!
\renewcommand{\dblfloatpagefraction}{0.9}       % require fuller float pages


\definecolor{tableaublue}{rgb}{0.44,0.62,0.81}
\definecolor{tableauorange}{rgb}{0.9,0.55,0.25} % original tableau value: {1.,0.62,0.29}
\definecolor{tableaugreen}{rgb}{0.4,0.75,0.36}
\definecolor{tableaured}{rgb}{0.93,0.4,0.36}
\definecolor{tableaupurple}{rgb}{0.68,0.55,0.79}
\newcommand{\green}[1]{\textcolor{tableaugreen}{\sf\bfseries #1}}
\newcommand{\orange}[1]{\textcolor{tableauorange}{\sf\bfseries #1}}
\newcommand{\blue}[1]{\textcolor{tableaublue}{\sf\bfseries #1}}
\newcommand{\red}[1]{\textcolor{tableaured}{\sf\bfseries #1}}
%\newcommand{\purple}[1]{\textcolor{tableaupurple}{\sf\bfseries #1}}

% TINY BIT MORE MICRO STUFF
% Always include hyperref last
\usepackage[bookmarks=true,breaklinks=true,letterpaper=true,colorlinks,citecolor=blue,linkcolor=blue,urlcolor=blue]{hyperref}
